\documentclass{article}

\usepackage{amsmath, amsthm, amssymb, amsfonts}
\usepackage{thmtools}
\usepackage{graphicx}
\usepackage{setspace}
\usepackage{geometry}
\usepackage{float}
\usepackage{hyperref}
\usepackage[utf8]{inputenc}
\usepackage[english]{babel}
\usepackage{framed}
\usepackage[dvipsnames]{xcolor}
\usepackage{tcolorbox}
\usepackage{indentfirst}
\usepackage{tikz}
\usetikzlibrary{arrows.meta}

\colorlet{LightGray}{White!90!Periwinkle}
\colorlet{LightOrange}{Orange!15}
\colorlet{LightGreen}{Green!15}
\colorlet{LightBlue}{Cyan!15}

\newcommand{\HRule}[1]{\rule{\linewidth}{#1}}
\declaretheoremstyle[name=Theorem,]{thmsty}
\declaretheorem[style=thmsty,numberwithin=section]{theorem}
\tcolorboxenvironment{theorem}{colback=LightGray}

\declaretheoremstyle[name=Proposition,]{prosty}
\declaretheorem[style=prosty,numberlike=theorem]{proposition}
\tcolorboxenvironment{proposition}{colback=LightOrange}

\declaretheoremstyle[name=Lemma,]{lemsty}
\declaretheorem[style=lemsty,numberlike=theorem]{lemma}
\tcolorboxenvironment{lemma}{colback=LightGreen}

\declaretheoremstyle[name=Definition,]{defsty}
\declaretheorem[style=defsty,numberlike=theorem]{definition}
\tcolorboxenvironment{definition}{colback=LightGreen}

\declaretheoremstyle[name=Problem,]{probsty}
\declaretheorem[style=probsty,numberlike=section]{problem}
\tcolorboxenvironment{problem}{colback=LightBlue}


\newenvironment{solution}{\paragraph{Solution:}}{\hfill$\square$}

\setstretch{1.2}
\geometry{
    textheight=9in,
    textwidth=5.5in,
    top=1in,
    headheight=12pt,
    headsep=25pt,
    footskip=30pt
}

\newcommand{\norm}[1]{\left\lVert#1\right\rVert}

% ------------------------------------------------------------------------------

\begin{document}

% ------------------------------------------------------------------------------
% Cover Page and ToC
% ------------------------------------------------------------------------------

\title{ \normalsize \textsc{}
		\\ [2.0cm]
		\HRule{1.5pt} \\
		\LARGE \textbf{\uppercase{Putnam 2024 A}
        \HRule{2.0pt} \\ [0.6cm] \LARGE{Math Competitions} \vspace*{10\baselineskip}
		}
		}
\date{\today}
\author{\textbf{Author} \\ 
		David Zhang 
        }
\maketitle
\newpage
\begin{problem}
    Find all $n \in \mathbb{Z}^+$ such that there exist solutions to 
    \begin{align*}
        2a^n + 3b^n = 4c^n,
    \end{align*}
    where $a,b,c \in \mathbb{Z}^+$.
\end{problem}

\begin{problem}
    Find all polynomials $p(x)$ such that there exist polynomials $q(x)$ that satisfy the equation: 
    \begin{align*}
        p(p(x))-x = (p(x)-x)^2q(x).
    \end{align*}
\end{problem}

\begin{problem}
    Let $S$ be the set of bijections 
    \begin{align}
        T : \{1,2,3\} \times \{1,2, \dots, 2024\} \rightarrow \{1,2,\dots,6072\}
    \end{align}
    such that $T(1,j) < T(2,j) < T(3,j)$ for all $j \in \{1,2,\dots, 2024\}$ and $T(i,j) < T(i,j+1)$ 
    for all $i \in \{1,2,3\}$ and $j \in \{1,2,\dots, 2023\}$. Do there exist $a$ and $c$ in 
    $\{1,2,3\}$ and $b$ and $d$ in $\{1,2,\dots,2024\}$ such that the fraction 
    of elements in $T$ in $S$ for which $T(a,b) < T(c,d)$ is at least $1/3$ and at most $2/3$?
\end{problem}

\begin{problem}
    Find all primes $p>5$ for which there exists an integer $a$ and an integer $r$ satisfying $1 \le r \le p-1$
    with the following property: the sequence $1,a,a^2, \dots, a^{p-5}$ can be arranged to form a sequence 
    $b_0, b_1, \dots, b_{p-5}$ such that $b_n -b_{n-1}-r$ is divisible by $p$ for $1 \le n \le p-5$.
\end{problem}


\begin{problem}
    Let $\Omega$ be a circle of radius $9$ centered at the origin. Let $\Delta$ be a disk of radius $1$ with 
    center $(r,0)$, $r \in [0,8]$. For two points $P,Q$ on $\Omega$, find the $r$ that minimizes the 
    probability for $PQ$ to be in $\Delta$.
\end{problem}


\begin{problem}
    For some sufficiently small $x$, write 
    \begin{align*}
        \frac{1-3x-\sqrt{(3x-1)^2-8x}}{4} = \sum_{k=0}^{\infty} c_k x^k.
    \end{align*}
For some $n \in \mathbb{Z}^n$, let $A$ be the $n\times n$ matrix with $i,j$ entry 
$c_{i+j-1}$ for $i,j \in \{1,\dots, n\}$. Find the determinant of $A$.
\end{problem}




% ------------------------------------------------------------------------------


% ------------------------------------------------------------------------------
% Reference and Cited Works
% ------------------------------------------------------------------------------

% ------------------------------------------------------------------------------

\end{document}
