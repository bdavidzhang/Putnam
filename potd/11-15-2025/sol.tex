\documentclass[amssymb,pra,12pt,aps,notitlepage]{revtex4-1}
\usepackage{mathptmx,amsmath, multirow, graphicx}
\usepackage{parskip}
\begin{document}

\title{Putnam POTD \today}
\maketitle

\section{Problem}

(2020 A2) Let $k$ be a nonnegative integer. Evaluate
\[
\sum_{j=0}^k 2^{k-j} \binom{k+j}{j}.
\]

\section{Solution}

We first try out a few terms, let 
\[
f(k) = \sum_{j=0}^k 2^{k-j} \binom{k+j}{j}
\]
\begin{itemize}
    \item $f(1) = 4$
    \item $f(2) = 16$
    \item $f(3) = 64$
    \item $f(4) = 256$
\end{itemize}
so we suspect that $f(k) = 4^k = 2^{2k}$, 
then it suffices to show that 

\[
g(k) = \sum_{j=0}^k 2^{-j} \binom{k+j}{j} = 2^{k}
\]

The few previous terms will be our base cases. Now 
for the inductive case, assume that 
$g(k) = 2^k$, consider $g(k+1)$:
\begin{align*}
    g(k+1) &= \sum_{j=0}^{k+1} 2^{-j} \binom{k+j+1}{j}\\
    &=  \sum_{j=0}^{k+1} 2^{-j} \binom{k+j+1}{k+1}\\
    &=  \sum_{j=0}^{k+1} 2^{-j} \left[\binom{k+j}{k} + \binom{k+j}{k+1}\right]\\
\end{align*}
here we use $\binom{a+1}{j+1} = \binom{a}{j} + \binom{a}{j+1}$ (known as the committee argument),
then 
\begin{align*}
    g(k+1)
    &=  \sum_{j=0}^{k+1} 2^{-j} \left[\binom{k+j}{k} + \binom{k+j}{k+1}\right]\\
    &=  \sum_{j=0}^{k+1} 2^{-j}\binom{k+j}{k} + \sum_{j=0}^{k+1} 2^{-j}\binom{k+j}{k+1}\\
    &=  \sum_{j=0}^{k} 2^{-j}\binom{k+j}{k} + 2^{-(k+1)}\binom{2k+1}{k+1} + \sum_{j=0}^{k+1} 2^{-j}\binom{k+j}{k+1}\\
    &=  g(k) + 2^{-(k+1)}\binom{2k+1}{k+1} + \sum_{j=0}^{k+1} 2^{-j}\binom{k+j}{k+1}\\
\end{align*}
Now we consider 
\begin{align*}
    L &=  \sum_{j=0}^{k+1} 2^{-j}\binom{k+j}{k+1} \\ 
    &=  \sum_{j=1}^{k+1} 2^{-j}\binom{k+j}{k+1} \\ 
    &=  \sum_{i=0}^{k} 2^{-(i+1)}\binom{k+i+1}{k+1} \\ 
    &=  \sum_{i=0}^{k} 2^{-(i+1)}\binom{k+i+1}{i} \\ 
\end{align*}
Note that this expression for $L$ looks a lot like $g(k+1)$, so we exploit it: 
\begin{align*}
    2L 
    &=  \sum_{i=0}^{k} 2^{-i}\binom{k+i+1}{i} \\ 
    &=  g(k+1) - 2^{-(k+1)} \binom{2k+2}{k+1} \\ 
\end{align*}
thus going back to the expression for $g(k)$, we have 
\begin{align*}
    g(k+1)
    &=  g(k) + 2^{-(k+1)}\binom{2k+1}{k+1} + L\\
    &=  g(k) + 2^{-(k+1)}\binom{2k+1}{k+1} + \frac{1}{2}\left(g(k+1) - 2^{-(k+1)} \binom{2k+2}{k+1}\right)\\
    \frac{1}{2}g(k+1) &= g(k) + 2^{-(k+1)}\binom{2k+1}{k+1} +  - 2^{-(k+2)} \binom{2k+2}{k+1}\\
     &= g(k) + 2^{-(k+2)}\left[2\binom{2k+1}{k+1} -   \binom{2k+2}{k+1}\right]\\
     &= g(k) + 2^{-(k+2)}\left[\binom{2k+1}{k+1}+\binom{2k+1}{k} -  \left(\binom{2k+1}{k}+\binom{2k+1}{k+1}\right)\right]\\
     &= g(k) + 2^{-(k+2)}\left[\binom{2k+1}{k+1}  - \binom{2k+1}{k+1}\right]\\
     &= g(k) + 0 = g(k)\\
\end{align*}
thus $g(k+1) = 2g(k) = 2^{k+1}$, and we are done.







\end{document}