\documentclass[amssymb,pra,12pt,aps,notitlepage]{revtex4-1}
\usepackage{mathptmx,amsmath, multirow, graphicx}
\usepackage{parskip}
\begin{document}

\title{Putnam POTD \today}
\maketitle

\section{Problem}
(1972 B3) Let $G$ be a group and let $g, h \in G$ satisfy
\[
ghg = hg^2h, \quad g^3 = 1,
\]
and suppose $h^m = 1$ for some odd integer $m$. Prove that $h=1$.

\section{(Wrong) Solution}
Rewrite the given equation as 
\[
ghg = hg^{-1}h
\]

\textbf{Please let me know how to show this and when can I make these substitutions!} We assume without proof that 
\[
g^{-1}hg^{-1} = hgh
\]
then we can claim:
\begin{align*}
    hghg &= h^2g^{-1}h \\ 
    g^{-1}h &= h^2g^{-1}h \\ 
    g^{-1} &= h^2g^{-1} \\ 
    1 &= h^2 \\ 
\end{align*}

and since $h^m = 1$ for some odd $m$, $h^{2n+1} = 1$ and thus 
$h = 1$.

\section{Solution}
Let's work backwards:
If we could show $gh^2 = h^2g$, then
\begin{align*}
    h^2 &= g^{-1}h^2 g\\
    h^{2k} &= g^{-1}h^{2k} g \quad \forall k \in \mathbb{Z}^{+}\\
\end{align*}
If $h^{2n+1} = 1$, let $k = n+1$, then 
\begin{align*}
    h^{2n+2} &= g^{-1}h^{2n+2} g \\
    h &= g^{-1}hg\\
    gh &= hg\\
    ghg &= hg^{-1}\\
    g^{-1}h &= hg^{-1}h\\
    h &= 1.
\end{align*}

Now we show $h^2g = gh^2$, to see this:
\[
gh^2 = ghg g^{-1}h = hg^{-1}hg^{-1}h = hg^{-1}ghg = h^2g,
\]
and we are done.

\section{Remark}
To be frank, I am currently not great with solving these problems, and 
I did not come up with the idea to this solution.

The solution wrote:
With hindsight, the correct approach is probably to work systematically through a set of increasingly complex expressions, trying to simplify them more than one way in order to get a new relation.



\end{document}