\documentclass[amssymb,pra,12pt,aps,notitlepage]{revtex4-1}
\usepackage{mathptmx,amsmath, multirow, graphicx}
\usepackage{parskip}
\usepackage{hyperref}
\begin{document}

\title{Putnam POTD \today}
\maketitle

\section{Problem}
\subsection{Problem 1}
Evaluate
\[ \int_{2}^{4} \frac{\sqrt{\ln(9-x)}}{\sqrt{\ln(9-x)} + \sqrt{\ln(x+3)}} \, dx \]
\subsection{Problem 2}
(A3, 1980 Putnam)
\[ \int_{0}^{\pi/2} \frac{dx}{1 + (\tan x)^{\sqrt{2}}} \]
\subsection{Problem 3}
Evaluate
\[ \int_{0}^{\pi/2} \ln(\sin x) \, dx \]
\section{Solution}

All three problems revolve around the same integral property:
for any function $f(x)$, 
\[
    \int_{0}^{a} f(x) dx = \int_{a}^0 f(a-x) (-dx) = \int_0^{a} f(a-x) dx.
\]

Which is called the \textbf{King property} for integrals because
it tends to be the most prominant of all integral properties.

\subsection{Problem 1}
Observe that $-3 \le x \le 9$, so the center is $x = 3$,
substitute $y = x-3$ to center around $0$, then the integral is 

\[ I = \int_{-1}^{1} \frac{\sqrt{\ln(6-y)}}{\sqrt{\ln(6-y)} + \sqrt{\ln(6+y)}} \, dy \]
making the substitution $u = -y$, 
\[ I = J = \int_{-1}^{1} \frac{\sqrt{\ln(6+u)}}{\sqrt{\ln(6+u)} + \sqrt{\ln(6-u)}} \, du \]
then 
\[I + J = 2I = \int_{-1}^{1} \frac{\sqrt{\ln(6-y)}+\sqrt{\ln(6+y)}}{\sqrt{\ln(6-y)} + \sqrt{\ln(6+y)}} \, dy = 2\]
so $I = 1$.

\subsection{Problem 2}
Let
\[ I = \int_{0}^{\pi/2} \frac{dx}{1 + (\tan x)^{\sqrt{2}}} \]
then using the King property
\[ I = \int_{0}^{\pi/2} \frac{dx}{1 + (\tan (\frac{\pi}{2} - x))^{\sqrt{2}}} \]
and since $\tan (\frac{\pi}{2} - x) = \cot(x)$,
\[ I = \int_{0}^{\pi/2} \frac{dx}{1 + (\cot (\frac{\pi}{2} - x))^{\sqrt{2}}} \]
then 
\[I + I = 2I = \int_{0}^{\pi/2} \frac{1}{1 + (\tan (\frac{\pi}{2} - x))^{\sqrt{2}}}+ \frac{1}{1 + (\cot (\frac{\pi}{2} - x))^{\sqrt{2}}} dx = \frac{\pi}{2}.\]
Thus $I = \frac{\pi}{4}$.


\subsection{Problem 3}
Let 
\[ I = \int_{0}^{\pi/2} \ln(\sin x) \, dx \]
then using King's property
\[ I = \int_{0}^{\pi/2} \ln(\sin (\frac{\pi}{2} - x)) \, dx  = \int_{0}^{\pi/2} \ln(\cos (x)) \, dx \]
thus 
\begin{align*}
    2I &=  \int_{0}^{\pi/2} \ln(\sin x) + \ln(\cos x) \, dx\\
    &=  \int_{0}^{\pi/2} \ln(\sin{2x} / 2) \, dx\\
    &=  \int_{0}^{\pi/2} \ln(\sin{2x}) - \ln{2} \, dx\\
    &=  \int_{0}^{\pi/2} \ln(\sin{2x})\, dx - \ln{2}\frac{\pi}{2}\\
\end{align*}
Now let $u = 2x$, then 
\begin{align*}
    \int_{0}^{\pi/2} \ln(\sin{2x})dx &= \int_{0}^{\pi} \ln(\sin{u}) du/2
\end{align*}
since $\sin{x}$ is symmetric about $\frac{\pi}{2}$ on 
$[0,\pi]$, $\ln(\sin{x})$ is also symmetric about 
$\frac{\pi}{2}$, then 
\begin{align*}
    \int_{0}^{\pi/2} \ln(\sin{2x})dx &= \int_{0}^{\pi/2} \ln(\sin{u}) du = I
\end{align*}
and once we substitute back to the given expression, 

\begin{align*}
    2I &= \int_{0}^{\pi/2} \ln(\sin{2x})\, dx - \ln{2}\frac{\pi}{2}\\
     &= I - \ln{2}\frac{\pi}{2}\\
\end{align*}
and thus $I = - \ln{2}\frac{\pi}{2}$.


\section{Remarks}
For more practice problems, see 
\href{https://static1.squarespace.com/static/59471704440243a2a0adc0e0/t/5ce06f9ad7031b00018e7e00/1558212507453/king+property.pdf}{practice problems}.

The name "King Property" (or "King's Rule") is widely believed to have been popularized in India by Vinod Kumar Bansal, a famous teacher for the JEE (Joint Entrance Examination) competitive exams.

Here's why it got the name:

It's the "King" of Integral Properties: In the context of solving tricky definite integrals for competitions, this property is considered the most powerful and most important one to use.

It "Rules" Over Many Problems: Just like the problem you shared, this property can elegantly solve integrals that look almost impossible at first glance. It's the "go-to" rule that "conquers" these problems.

Part of a Set: This teacher also reportedly named other, related properties the "Queen's Rule" and "Jack's Rule," treating this one as the "King" of the set.

The King PropertyThis is the one you just used. It's the most general and powerful, which is why it's the "King."Formula: $\int_a^b f(x) \, dx = \int_a^b f(a+b-x) \, dx$Common Case: $\int_0^a f(x) \, dx = \int_0^a f(a-x) \, dx$


The Queen PropertyThe "Queen Property" is a variation that deals with integrals from $0$ to $2a$. It's used to "fold" the interval in half.Formula: $\int_0^{2a} f(x) \, dx = \int_0^a f(x) \, dx + \int_0^a f(2a-x) \, dx$This rule becomes especially powerful in two specific cases:If $f(2a-x) = f(x)$ (the function is symmetric about $x=a$), then:$\int_0^{2a} f(x) \, dx = 2 \int_0^a f(x) \, dx$If $f(2a-x) = -f(x)$ (the function is anti-symmetric about $x=a$), then:$\int_0^{2a} f(x) \, dx = 0$

The Jack PropertyThe "Jack Property" is the most common and basic rule, usually taught first. It deals with even and odd functions integrated over a symmetric interval, like $[-a, a]$.If $f(x)$ is an EVEN function (i.e., $f(-x) = f(x)$), then:$\int_{-a}^a f(x) \, dx = 2 \int_0^a f(x) \, dx$(The area on the left side is identical to the area on the right.)If $f(x)$ is an ODD function (i.e., $f(-x) = -f(x)$), then:$\int_{-a}^a f(x) \, dx = 0$(The area on the left is the exact negative of the area on the right, so they cancel out.)


\end{document}