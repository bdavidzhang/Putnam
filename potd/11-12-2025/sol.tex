\documentclass[amssymb,pra,12pt,aps,notitlepage]{revtex4-1}
\usepackage{mathptmx,amsmath, multirow, graphicx}
\usepackage{parskip}
\begin{document}

\title{Putnam POTD \today}
\maketitle

\section{Problem}
(2017 A2) Let $Q_0(x) = 1$, $Q_1(x) = x$, and
\[
Q_n(x) = \frac{(Q_{n-1}(x))^2 - 1}{Q_{n-2}(x)}
\]
for all $n \geq 2$. Show that, whenever $n$ is a positive integer, $Q_n(x)$ is equal to a polynomial with integer coefficients.

\section{Solution}
Given the form of the question, it is inviting to use induction. 
However, and I must point out that this could just be my algebraic skills, but 
the given form 
\[
Q_n(x) = \frac{(Q_{n-1}(x))^2 - 1}{Q_{n-2}(x)}
\]
does not have a directly good algebraic form to make proofs pertaining our claim 
that $Q_n(x) \in \mathbb{Z}[x]$. 

Instead, we may be motivated to find an alternative form of the expression:
Let's observe the terms:

\begin{itemize}
    \item $Q_0(x) = 1$
    \item $Q_1(x) = x$
    \item $Q_2(x) = x^2 - 1$
    \item $Q_3(x) = x^3 - 2$
    \item $Q_4(x) = x^4 - 3x^2 + 1$
\end{itemize}
Do you see 
\[
    Q_n(x) = xQ_{n-1}(x) - Q_{n-2}?
\]

Now we need to that this is indeed the case. Since we are given 
\[
Q_n(x) = \frac{(Q_{n-1}(x))^2 - 1}{Q_{n-2}(x)},
\]
the justification lies in this expression leading to the linear recurrence. 

We first assume via induction that the linear expression is valid up to $Q_{n-1}$, 
then consider 
\begin{align*}
    Q_n(x) &= \frac{(Q_{n-1}(x))^2 - 1}{Q_{n-2}(x)}\\
    &= \frac{x^2Q_{n-2}^2  - 2xQ_{n-2}Q_{n-3}+ Q_{n-3}^2-1}{Q_{n-2}(x)}\\
    &= \frac{x^2Q_{n-2}^2  - 2xQ_{n-2}Q_{n-3} + Q_{n-2}Q_{n-4}}{Q_{n-2}(x)}\\
    &= {x^2Q_{n-2} - 2xQ_{n-3} + Q_{n-4}}\\
    &= {x^2Q_{n-2} - xQ_{n-3} - (xQ_{n-3} - Q_{n-4})}\\
    &= {x^2Q_{n-2} - xQ_{n-3} - Q_{n-2}}\\
    &= {x(xQ_{n-2} - Q_{n-3}) - Q_{n-2}}\\
    &= {xQ_{n-1} - Q_{n-2}},
\end{align*}
and we have derived the linear recurrence!


Now assume again via induction that for all natural numbers $i < m$, 
$Q_i \in \mathbb{Z}[x]$, then 
\[Q_m = xQ_{m-1} - Q_{m-2}\]
since $\mathbb{Z}[x]$ is closed under addition and multiplication, 
$Q_m$ is an integer. 


\section{Observations}
The expression 
\(
Q_n(x) = \frac{(Q_{n-1}(x))^2 - 1}{Q_{n-2}(x)},
\)
seems quite familar. I remember an AIME problem with 
a similar setting, but to find the $2021$th (I made that up)
term, given the inital expresion involving a 
ratio between several terms. The most common idea 
was to write down a few terms to gain some intution. 

The Core Idea: Linear vs. Quadratic RecurrenceThe problem's "trick" is that a messy, non-linear (quadratic) recurrence relation is equivalent to a simple, linear one.The Given Recurrence (Quadratic):$Q_n(x) = \frac{(Q_{n-1}(x))^2 - 1}{Q_{n-2}(x)}$This is hard to work with.

If you rearrange it, you get
\[Q_n(x) Q_{n-2}(x) = (Q_{n-1}(x))^2 - 1\]
The "Hidden" Recurrence (Linear):
\[Q_n(x) = x Q_{n-1}(x) - Q_{n-2}(x)\]

This is very easy to work with. If you assume this recurrence is true, the proof is trivial:Base Cases: $Q_0 = 1$ and $Q_1 = x$ are polynomials with integer coefficients.Inductive Step: Assume $Q_{n-1}$ and $Q_{n-2}$ are polynomials with integer coefficients. Then $Q_n = x \cdot Q_{n-1} - Q_{n-2}$ is just the product, sum, and difference of such polynomials, which must also be a polynomial with integer coefficients.

\subsection{Remark 1: A motivation for the problem}
Remark 1: The "Real" Identity (Chebyshev Polynomials)This is the most important piece of background. 
This problem is a disguised version of a standard identity for Chebyshev Polynomials.Chebyshev Polynomials of the Second Kind ($U_n(y)$)
 are defined by the exact same linear recurrence, just with a slightly different variable:
\[U_0(y) = 1, U_1(y) = 2y, \dots, U_n(y) = 2y \cdot U_{n-1}(y) - U_{n-2}(y)\]

The Connection: If you make the substitution $x = 2y$ (or $y = x/2$), our polynomials $Q_n(x)$ are identical to $U_n(y)$. 
Therefore, $Q_n(x) = U_n(x/2)$.The original problem's identity \[Q_n(x) Q_{n-2}(x) = (Q_{n-1}(x))^2 - 1\]
 is just a famous, standard identity for Chebyshev polynomials, which is 
 
\[U_n(y) U_{n-2}(y) = (U_{n-1}(y))^2 - 1.\]
This identity is usually proven using the trigonometric definition $U_n(\cos \theta) = \frac{\sin((n+1)\theta)}{\sin \theta}$, where it becomes a simple trig identity.


\subsection{Remark 2: Matrix Proof \& Cassini's Identity}
This remark connects the problem to another famous example: the Fibonacci sequence.

Fibonacci Recurrence: $F_n = F_{n-1} + F_{n-2}$

Cassini's Identity: $F_{n+1} F_{n-1} - F_n^2 = (-1)^n$


Notice the pattern: a second-order linear recurrence ($F_n = ...$) leads to a quadratic identity (Cassini's). 

Our problem is the same:

Linear Recurrence: $Q_n = x Q_{n-1} - Q_{n-2}$

Quadratic Identity: $Q_n Q_{n-2} - Q_{n-1}^2 = -1$, or $(Q_{n-1})^2 - Q_n Q_{n-2} = 1$.

The matrix proof  is the standard way to prove all of these identities.The matrix $T = \begin{pmatrix} x & -1 \\ 1 & 0 \end{pmatrix}$ is the "transition matrix" for the linear recurrence.

The identity $T M_n = M_{n+1}$ just says "applying the transition matrix to the state at step $n$ gives the state at step $n+1$."

The core of the proof is that $\det(M_{n+1}) = \det(T M_n) = \det(T) \det(M_n)$.

Since $\det(T) = (x)(0) - (-1)(1) = 1$, the determinant never changes!

\[\det(M_n) = \det(M_{n-1}) = \dots = \det(M_1). \quad \det(M_1) = \begin{vmatrix} P_0 & P_1 \\ P_{-1} & P_0 \end{vmatrix} = \begin{vmatrix} 1 & x \\ 0 & 1 \end{vmatrix} = 1.\]


Therefore, $\det(M_n) = (P_{n-1})^2 - P_n P_{n-2} = 1$ for all $n$.

This is a beautiful and general technique. For the Fibonacci sequence, the transition matrix is $\begin{pmatrix} 1 & 1 \\ 1 & 0 \end{pmatrix}$, which has a determinant of -1. This is precisely why the $F_n$ identity has $(-1)^n$ in it, while the $Q_n$ identity has $1^n = 1$.

As mentioned, in general, \textbf{a second-order linear recurrence ($F_n = ...$) leads to a quadratic identity (Cassini's). }

\subsection{Remark 3: Combinatorial Interpretation}

The third way to understand these polynomials, this time through combinatorics.

They define a new set of polynomials, $R_n(x)$, called Fibonacci Polynomials (though this name is used for a few different things).



Definition: $R_n(x)$ is the generating function for the number of ways to tile a $1 \times n$ board with $1 \times 1$ squares (which count as "$x$") and $1 \times 2$ dominoes (which count as "1").


Example $R_3(x)$:Tile (1,1,1): $x \cdot x \cdot x = x^3$Tile (1, 2): $x \cdot 1 = x$Tile (2, 1): $1 \cdot x = x$Total: $R_3(x) = x^3 + 2x$\\

Recurrence for $R_n(x)$: To tile a $1 \times n$ board, you can either:Start with a $1 \times 1$ square (cost: $x$) and then tile the remaining $1 \times (n-1)$ board. This gives $x \cdot R_{n-1}(x)$.Start with a $1 \times 2$ domino (cost: $1$) and then tile the remaining $1 \times (n-2)$ board. This gives $1 \cdot R_{n-2}(x)$.

This gives the linear recurrence \[R_n(x) = x R_{n-1}(x) + R_{n-2}(x).\]

This is almost our $Q_n$ recurrence, but with a + sign instead of a -. 

We can then connect them with a change of variables:
\[Q_n(x) = i^{-n} R_n(ix)\]
This algebraic substitution transforms one linear recurrence into the other. The complex combinatorial identity for $R_n$ is just another version of Cassini's identity, which, when transformed, becomes the identity for $Q_n$.


See related:

(1993 A2) Let $(x_n)_{n \geq 0}$ be a sequence of nonzero real numbers
such that $x_n^2 - x_{n-1}x_{n+1} = 1$ for $n=1,2,3,\dots$. Prove there
exists a real number $a$ such that $x_{n+1} = ax_n - x_{n-1}$ for all $n
\geq 1$.


\end{document}