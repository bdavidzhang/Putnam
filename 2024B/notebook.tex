\documentclass{article}

\usepackage{amsmath, amsthm, amssymb, amsfonts}
\usepackage{thmtools}
\usepackage{graphicx}
\usepackage{setspace}
\usepackage{geometry}
\usepackage{float}
\usepackage{hyperref}
\usepackage[utf8]{inputenc}
\usepackage[english]{babel}
\usepackage{framed}
\usepackage[dvipsnames]{xcolor}
\usepackage{tcolorbox}
\usepackage{indentfirst}
\usepackage{tikz}
\usetikzlibrary{arrows.meta}

\colorlet{LightGray}{White!90!Periwinkle}
\colorlet{LightOrange}{Orange!15}
\colorlet{LightGreen}{Green!15}
\colorlet{LightBlue}{Cyan!15}

\newcommand{\HRule}[1]{\rule{\linewidth}{#1}}
\declaretheoremstyle[name=Theorem,]{thmsty}
\declaretheorem[style=thmsty,numberwithin=section]{theorem}
\tcolorboxenvironment{theorem}{colback=LightGray}

\declaretheoremstyle[name=Proposition,]{prosty}
\declaretheorem[style=prosty,numberlike=theorem]{proposition}
\tcolorboxenvironment{proposition}{colback=LightOrange}

\declaretheoremstyle[name=Lemma,]{lemsty}
\declaretheorem[style=lemsty,numberlike=theorem]{lemma}
\tcolorboxenvironment{lemma}{colback=LightGreen}

\declaretheoremstyle[name=Definition,]{defsty}
\declaretheorem[style=defsty,numberlike=theorem]{definition}
\tcolorboxenvironment{definition}{colback=LightGreen}

\declaretheoremstyle[name=Problem,]{probsty}
\declaretheorem[style=probsty,numberlike=section]{problem}
\tcolorboxenvironment{problem}{colback=LightBlue}


\newenvironment{solution}{\paragraph{Solution:}}{\hfill$\square$}

\setstretch{1.2}
\geometry{
    textheight=9in,
    textwidth=5.5in,
    top=1in,
    headheight=12pt,
    headsep=25pt,
    footskip=30pt
}

\newcommand{\norm}[1]{\left\lVert#1\right\rVert}

% ------------------------------------------------------------------------------

\begin{document}

% ------------------------------------------------------------------------------
% Cover Page and ToC
% ------------------------------------------------------------------------------

\title{ \normalsize \textsc{}
		\\ [2.0cm]
		\HRule{1.5pt} \\
		\LARGE \textbf{\uppercase{Putnam 2024 A}
        \HRule{2.0pt} \\ [0.6cm] \LARGE{Math Competitions} \vspace*{10\baselineskip}
		}
		}
\date{\today}
\author{\textbf{Author} \\ 
		David Zhang 
        }
\maketitle
\newpage
\begin{problem}
    Let $n, k \in \mathbb{Z}^+$. The sequence in the $i$th row and $j$th column of an $n\times n$ grid contains 
    the numbe $i+j-k$. For which $n$ and $k$ is it possible to select $n$ squares from the grid, no two in the same 
    row or column, such that the numbers contained in the selected squares are $1,2,\dots, n$?
\end{problem}

\begin{problem}
    Two convex quadrilaterals are called partners if they have three verticies in common and they can be labelled 
    $ABCD$ and $ABCE$ so that $E$ is the reflection of $D$ across the perpendicular bisector of the diagonal $AC$. 
    Is there an infinite sequence of convex quadrilaterals such that each quadrilateral is a partner of its successor 
    and no two elements of the sequence are congruent?
\end{problem}

\begin{problem}
    Let $r_n$ be the smallet positive solution to $\tan{x} = x$, where the argument of tangent is in radians. Prove that 
    \begin{align}
        0 < r_{n+1} - r_n - \pi < \frac{1}{(n^2+n)\pi}
    \end{align}
    for $n \ge 1$.
\end{problem}

\begin{problem}
    Let $n$ be a positive integer. Set $a_{n,0} = 1$. For $k \ge 0$, choose an integer $m_{n,k}$ uniformly at random 
    from the set $\{1,\dots,n\}$ and let 
    $$
    a_{n,k+1} = 
    \begin{cases}
        a_{n,k} + 1 \mathrm{if} m_{n,k} > a_{n,k};\\
        a_{n,k}  \mathrm{if} m_{n,k} = a_{n,k};\\
        a_{n,k} - 1 \mathrm{if} m_{n,k} < a_{n,k}.\\
    \end{cases}
    $$
    Let $E(n)$ be the expected value of $a_{n,n}$. Determine $\lim_{n \to \infty} E(n)/n$.
\end{problem}


\begin{problem}
    Let $k$ and $m$ be integers. For a positive integer $n$, let $f(n)$ be the number of integer sequences $x_1, \dots, x_k, y_1, \dots, y_m, z$ 
    satisfying $1 \le x_1 \le \dots \le x_k \le z \le n$ and $1 \le y_1 \le \dots \le y_m \le z \le n$. Show that 
    $f(n)$ can be expressed as a polynomial in $n$ with nonegative coefficients.
\end{problem}


\begin{problem}
    For a real number $a$, let $F_a(x) = \sum_{n \ge 1} n^a e^{2n} x^{n^2}$ for $0 \le x < 1$. Find a real number $c$ such that 
    \begin{align}
        \lim{x \to 1^-} F_a(x) e^{-1/(1-x)} &= 0 \text{for all } a < c, \text{and }\\
        \lim{x \to 1^-} F_a(x) e^{-1/(1-x)} &= \infty \text{for all } a > c.
    \end{align}
\end{problem}




% ------------------------------------------------------------------------------


% ------------------------------------------------------------------------------
% Reference and Cited Works
% ------------------------------------------------------------------------------

% ------------------------------------------------------------------------------

\end{document}
